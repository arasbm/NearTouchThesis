\startchapter{Conclusions and Future Work}
\label{concl}

\section{Future Work}
The best models presented as Algorithms 1 and 2 can be used to design applications with near touch and bimanual interaction capabilities, which can be evaluated for usability in future studies.

In order to apply my results to a more general context in which the user can move around the display, one could study a user's position with respect to the interaction surface.
One approach that has been used for localizing presence of users near an interactive surface is proximity sensing \cite{Annett:2011:MPAMT}.
In this study I used a static angle of $60\degree$ and a fixed size display for the interface. 
Further research could measure the effects of changing the angle and size of the display for my presented models. In addition, future research should examine the effect of a crowded scene on gesture interactions, since this could change the way in which users grab and release objects.

Because grab and release actions did not produce a large difference in the proposed models, I excluded action type from the model for the center of action.
For other action types that involve different patterns in the trajectory of features, action type may need to be considered as a factor.
Therefore, future research could extend this model to cover other near touch actions besides grab and release.  
One could parameterize and detect other near touch actions such as flick, scoop, shake, throw, turn, twist, etc.
I intend to focus my future research on these types of actions as opposed to symbolic gestures, because of their intuitiveness and cross cultural meanings.

\section{Conclusions}
I have presented empirical findings for parameterizing grab and release actions using a customized near touch technology. 
I used these findings to define a method for locating the center of grab and release actions as well as an approach for distinguishing which hand was used for the interaction.
These empirical model data and near touch tracking system contributions provide new opportunities for natural and intuitive hand interactions with computing surfaces. 

All data from my study is available to download for anyone interested in joining
me to continue this research (https://github.com/arasbm/NearTouchThesis).

